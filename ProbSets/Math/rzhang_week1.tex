\documentclass[letterpaper,12pt]{article}
\usepackage{array}
\usepackage{threeparttable}
\usepackage{geometry}
\geometry{letterpaper,tmargin=1in,bmargin=1in,lmargin=1.25in,rmargin=1.25in}
\usepackage{fancyhdr,lastpage}
\pagestyle{fancy}
\lhead{}
\chead{}
\rhead{}
\lfoot{\footnotesize\textsl{OSM Lab, Summer 2017, Math PS \#1}}
\cfoot{}
\rfoot{\footnotesize\textsl{Page \thepage\ of \pageref{LastPage}}}
\renewcommand\headrulewidth{0pt}
\renewcommand\footrulewidth{0pt}
\usepackage[format=hang,font=normalsize,labelfont=bf]{caption}
\usepackage{amsmath}
\usepackage{amssymb}
\usepackage{amsthm}
\usepackage{natbib}
\usepackage{upgreek}
\usepackage{setspace}
\usepackage{float,color}
\usepackage[pdftex]{graphicx}
\usepackage{hyperref}
\usepackage{mathrsfs}
\usepackage{dsfont}
\hypersetup{colorlinks,linkcolor=red,urlcolor=blue,citecolor=red}
\theoremstyle{definition}
\newtheorem{theorem}{Theorem}
\newtheorem{acknowledgement}[theorem]{Acknowledgement}
\newtheorem{algorithm}[theorem]{Algorithm}
\newtheorem{axiom}[theorem]{Axiom}
\newtheorem{case}[theorem]{Case}
\newtheorem{claim}[theorem]{Claim}
\newtheorem{conclusion}[theorem]{Conclusion}
\newtheorem{condition}[theorem]{Condition}
\newtheorem{conjecture}[theorem]{Conjecture}
\newtheorem{corollary}[theorem]{Corollary}
\newtheorem{criterion}[theorem]{Criterion}
\newtheorem{definition}[theorem]{Definition}
\newtheorem{derivation}{Derivation} % Number derivations on their own
\newtheorem{example}[theorem]{Example}
\newtheorem{exercise}[theorem]{Exercise}
\newtheorem{lemma}[theorem]{Lemma}
\newtheorem{notation}[theorem]{Notation}
\newtheorem{problem}[theorem]{Problem}
\newtheorem{proposition}{Proposition} % Number propositions on their own
\newtheorem{remark}[theorem]{Remark}
\newtheorem{solution}[theorem]{Solution}
\newtheorem{summary}[theorem]{Summary}
%\numberwithin{equation}{section}
\bibliographystyle{aer}
\newcommand\ve{\varepsilon}
\newcommand\boldline{\arrayrulewidth{1pt}\hline}

\begin{document}

\title{Math Problem Set 1 \\
Open Source Macroeconomics Laboratory Boot Camp}
\author{Ruby Zhang}
\maketitle

\begin{enumerate}
  \item Exercises from the book:
    \begin{enumerate}
    \item[3.6]
    Since $A \in \mathscr{F}$, which is the power set of $\Omega$, then $A \subset \Omega = \cup_{i \in I} B_i$. Therefore, $A = \cup_{i \in I} A \cap B_i$. By the definition of a probability space, $P$ is countably additive on $\mathscr{F}$ so $P(\cup_{i \in I} A \cap B_i) = P(A) = \sum_{i \in I} P(A \cap B_i)$.
    \item[3.8]
    First, we will prove that if $E_1$ and $E_2$ are independent events, then $E_1^c$ and $E_2^c$ are also independent events. Note that:
      \begin{align*}
        P(E_1^c \cap E_2^c) &= P((E_1 \cup E_2)^c) \\
        & = 1 - (P(E_1) + P(E_2) - P(E_1 \cap E_2)) \\
        &= 1 - P(E_1) - P(E_2) + P(E_1) \times P(E_2) \\
        & = (1 - P(E_1))\times(1-P(E_2)) \\
        & = P(E_1^c) \times P(E_2^c)
      \end{align*}
    From the rule of unions and intersections, we have that $(\cup_{k=1}^n E_k)^c = \cap_{k=1}^n E_k^c$. In addition, since $\{E_k\}_{k=1}^n$ is a collection of independent events, so by extending the previous proof inductively we have that $\{E_k^c\}_{k=1}^n$ is also a collection of independent events. Then we have that:
      \begin{align*}
        P(\cup_{k=1}^n E_k) &= 1 - P((\cup_{k=1}^n E_k)^c) \\
        &= 1 - P(\cap_{k=1}^n E_k^c) \\
        &= 1 - \prod_{k=1}^n P(E_k^c) \\
        &= 1 - \prod_{k=1}^n (1 - P(E_k))
      \end{align*}
    \item[3.11]
      From Bayes' Rule, we have that
      \begin{align*}
        P(s=\text{ crime}|s\text{ tested }+) &= \frac{P(s\text{ tested }+|s=\text{ crime})P(s=\text{ crime})}{P(s\text{ tested }+)} \\
        &= \frac{1 \times \frac{1}{250,000,000}}{\frac{1}{3,000,000}} \\
        &= \frac{3}{250}
      \end{align*}
    \item[3.12]
      Without loss of generality, suppose the contestant picked door $A_1$ and Monty opened door $A_2$, which contains a goat. We want to show that the contestant is better off picking door $A_3$. Since the contestant chose the first door with no prior information, then $P(A_1)=1/3$. Therefore, $P(A_2 \cup A_3)=2/3$. However, if $A_2$ contains a goat, then $P(A_3)=0$ and we know that $P(A_2 \cap A_3)$ since the car cannot be behind both. Thus, $P(A_2 \cup A_3)=P(A_3)=2/3$. The contestant would have double the chance of winning if they switched doors versus sticking with their original decision.

      In a similar situation with 10 doors, you would have a $1/10$ probability of winning if you stuck with your original decision, but you have a $9/10$ probability of winning if you switched to the remaining door.
    \item[3.16]
      We want to show that $E[(X-\mu)^2] = E[X^2] - \mu^2$. From the definition of variance and the fact that expectation is the weighted average and is thus additive, we have that:
        \begin{align*}
          Var[X] &= E[(X-\mu)^2] \\
          &= E[(X-E[X])^2] \\
          &= E[X^2 - 2XE[X] + E[X]^2] \\
          &= E[X^2] - E[2XE[X]] + E[E[X]^2] \\
          &= E[X^2] - 2E[X]^2 + E[X^2] \text{ since $E[X]^2$ is a constant} \\
          &= E[X^2] - E[X]^2 \\
          &= E[X^2] - \mu^2
        \end{align*}
    \item[3.33]
      For a binomial random variable $B$, we have that $B = \sum_{i=1}^n B_i$, where all $B_i$'s are independently Bernoulli distributed random variables such that $E[B_i] = p$ and $Var(B_i) = p(1-p)$. Thus, using the weak law of large numbers, we have that for all $\epsilon > 0$,
        \begin{equation*}
          P(|\frac{\sum_{i=1}^n B_i}{n} - p| \geq \epsilon) = P(|\frac{B}{n} - p| \geq \epsilon) \leq \frac{p(1-p)}{n\epsilon^2}
        \end{equation*}
    \item[3.36]
      Let $S_n = \sum_{i=1}^{6242} X_i$, where $X_i$ is the probability that student $i$ will enroll at the school, so $X_i$'s are independently Bernoulli distributed random variables. Thus, $E[X_i]=\mu = 0.801$ and $Var[X_i] = 0.801 \times 0.199 = 0.1594$. We want to estimate $P(S_n \geq 5500) = 1 - P(S_n \leq 5500)$. Using the Central Limit Theorem, we have the estimation that:
        \begin{align*}
          P(\frac{S_n - n\mu}{\sigma\sqrt{n}} \leq y) &= P(S_n \leq \sigma\sqrt{n}y + n\mu) \\
          & \approx \frac{1}{\sqrt{2\pi}}\int_{-\infty}^y e^{\frac{-x^2}{2}}dx
        \end{align*}
      We can calculate $y$ as follows:
        \begin{align*}
          \sigma\sqrt{n}y + n\mu &= 5500 \\
          y &= \frac{5500 - 6242\times 0.801}{\sqrt{0.1594\times 6242}} \\
          &= 15.8563
        \end{align*}
      Computing the integral using wolfram alpha, we thus have that:
        \begin{equation*}
          P(S_n \geq 5500) = 1 - P(S_n \leq 5500) = 1 - 1 = 0
        \end{equation*}
    \end{enumerate}
  \item
    \begin{enumerate}
      \item
        Suppose we toss two independent coins. Let A be the event that coin 1 is heads and B be the event that coin 2 is heads. Let C be the event if there is exactly one head amongst the two coin tosses. The four possible outcomes are: $\Omega = \{HH, TT, HT, TH\}$, thus we have that $P(B)=P(C)=P(A)=\frac{1}{2}$. Note the following observations:
          \begin{align*}
            P(A \cap B) & = P(\{HH\}) = \frac{1}{4} = P(A)P(B) \\
            P(A \cap C) & = P(\{HT\}) = \frac{1}{4} = P(A)P(C) \\
            P(B \cap C) & = P(\{TH\}) = \frac{1}{4} = P(B)P(C) \\
            P(A \cap B \cap C) &= P(\{\}) = 0 \neq P(A)P(B)P(C)
          \end{align*}
      \item
        Suppose we have a fair 8-sided die. Let $B = \{1, 2, 3, 4\}, C = \{1,5,6,7\}$ and $A = \{1, 4, 7, 8\}$ denote events that consist of possible die rolls. Then $P(B)=P(C)=P(A)=\frac{1}{2}$. Note the following observations:
          \begin{align*}
            P(A \cap B) & = P(\{1,4\}) = \frac{1}{4} = P(A)P(B) \\
            P(A \cap C) & = P(\{1,7\}) = \frac{1}{4} = P(A)P(C) \\
            P(A \cap B \cap C) & = P(\{1\}) = \frac{1}{8} = P(A)P(B)P(C) \\
            P(B \cap C) &= P(\{1\}) = \frac{1}{8} \neq P(B)P(C)
          \end{align*}
    \end{enumerate}
  \item
    To prove that Benson's Law is a well-defined discrete probability distribution, we must show that the probability of the entire space of outcomes is 1:
      \begin{align*}
        P(\Omega) &= \sum_{d = 1}^9 \log_{10} (1+\frac{1}{d}) \\
        &= \log_{10} (\sum_{d = 1}^9 \frac{d+1}{d}) \text{ this is a telescoping sum} \\
        &= \log_{10} 10 \\
        &= 1
      \end{align*}
  \item
    \begin{enumerate}
      \item
        The probability that the person wins $\$2^n$ is if they flip $n-1$ heads in a row then tails on the $n$th flip. The probability of that happening is $1/2^n$. Thus, for any given winning $x_n = \$2^n$, $p_nx_n = 1$. Since $n = \mathds{N}$, then $E[X] = \sum_{n = 1}^\infty p_nx_n = \sum_{n = 1}^\infty 1 = + \infty$.
      \item
        Since the player has log utility, for a given winning $x_n = \$2^n$, the utility $u_n = n\log{2}$. The probability has not changed, thus we have that $E[\log{X}] = \sum_{n = 1}^\infty p_nu_n = \log{2} \sum_{n = 1}^\infty \frac{n}{2^n} = 2\log{2}$, which was found using Wolfram Alpha.
    \end{enumerate}
  \item
    If the U.S. investor invests a dollar in the Swiss currency instead of U.S. currency this year, there is $\frac{1}{2}$ chance that next year he will receive $\$1.25$ USD. and $\frac{1}{2}$ chance that next year he will receive $0.80$ USD. Thus, the expected value is $\frac{1.25 + 0.8}{2} = \$1.025$ USD. Thus, the U.S. investor should invest in Swiss currency.
    If the Swiss investor invests a dollar in U.S. currency instead of Swiss currency, there is $\frac{1}{2}$ chance that next year he will receive $\$1.25$ CHF and $\frac{1}{2}$ chance that next year he will receive $0.80$ CHF. Thus, the expected value is $\frac{1.25 + 0.8}{2} = \$1.025$ CHF. Similarly, the Swiss investor should invest in U.S. currency.
  \item
    \begin{enumerate}
      Note that for all the following questions, the random variable are all valid since they are continuous (and thus map $[0,1]$ to $\mathds{R}$).
      \item
        Let the probability density function of X be the following:
          \[
            f_X(x) =
            \begin{cases}
              \frac{2}{x^3} & x \geq 1 \\
              0 & otherwise
            \end{cases}
          \]
        Note that $\int_{-\infty}^{\infty} f_X(x) dx = \int_1^{\infty} \frac{2}{x^3}dx= 1$, thus this is a valid continuous random variable. Note that:
          \begin{align*}
            E[X] &= \int_1^{\infty} x\frac{2}{x^3} dx \\
            &= [-\frac{2}{x}]_1^{\infty} \\
            &= 2 \\
            E[X^2] &=\int_1^{\infty} x^2\frac{2}{x^3} dx \\
            &= [2\log{x}]_1^{\infty} \\
            &= \infty
          \end{align*}
      \item
      Let $X \sim U(0,1)$ and $Y \sim exp(1.99)$. Thus, $E[X]=\frac{1}{2} < E[Y]=\frac{1}{1.99}$. Now we must show that $P(X > Y) > 1/2$:
        \begin{align*}
          P(X > Y) &= \int_{-\infty}^{\infty} P(X >Y|Y = y)f_Y(y)dy \\
          &=  \int_{-\infty}^{\infty} P(X >y)f_Y(y)dy \\
          &= \int_{-\infty}^{\infty} (1 - F_X(y))f_Y(y)dy \\
          &= 1-\int_{-\infty}^{\infty}F_X(y)f_Y(y)dy \\
          &= 1-(\int_0^1 y 1.99e^{-1.99y}dy + \int_1^{\infty} 1.99e^{-1.99y}dy)\\
          &= 0.566
        \end{align*}
      \item
      Let $X \sim U(-1,1), Y \sim U(-2,2)$, and $Z \sim U(-3,3)$. From part b), we have that $P(X > Y) = 1-\int_{-\infty}^{\infty}F_X(y)f_Y(y)dy$. Using the pdfs and cdfs of the uniform distribution, we have that:
        \begin{align*}
          P(X > Y) &= 1 - \frac{1}{4} - \frac{1}{4} = \frac{1}{2} \\
          P(X > Z) &= 1 - \frac{1}{6} - \frac{1}{3} = \frac{1}{2} \\
          P(Z > Y) &= 1 - \frac{1}{3} - \frac{1}{6} = \frac{1}{2} \\
        \end{align*}
      Thus, $P(X > Y)P(X > Z)P(Z > Y) = \frac{1}{8} > 0$ and $E[X] = E[Y] = E[Z] = 0$.
    \end{enumerate}
  \item
    \begin{enumerate}
      \item
      If $Z=1$, then $Y = X \sim N(0,1)$. If $Z=-1$, then $Y = -X \sim N(0,1)$ due to the symmetry of $N(0,1)$ about 0. Then for a given $y$, we have that:
        \begin{align*}
          P(Y < y) &= P(Y < y | Z = 1)P(Z=1) + P(Y < y | Z = -1)P(Z=-1)\\
          &= \frac{P(X < y) + P(-X < y)}{2} \\
          &= \frac{P(X < y) + P(X > -y)}{2} \\
          &= P(X < y) \text{ due to symmetry of } N(0,1)
        \end{align*}
        Since this holds for all $y$, then $Y \sim N(0,1)$.
      \item
      For any given value $X=x$, we have that $Y \in {-x, x}$, thus $|X| = |Y|$ with certainty. $P(|X|=|Y|)=1$ then follows.
      \item
      Note that $f_{XY}(1, 2) = 0$ since $Y \in {-1,1}$. However, $f_X(1)f_Y(2) = f_X(1)f_X(2) \neq 0$. Thus, $X$ and $Y$ are not independent.
      \item
      We have that $Cov[X,Y] = E[XY] - E[X]E[Y] = E[XY] = E[X^2Z] = \frac{E[X^2] - E[X^2]}{2} = 0$.
      \item
      As seen from the previous parts, $X, Y$ are normally distributed random variables with covariance zero, but are dependent. Therefore, the statement is false since there is a counterexample.
    \end{enumerate}
  \item
    First, let us state the cumulative distribution function of $X_i \sim U[0,1]$:
      \[
        F_{X_i}(x) =
        \begin{cases}
          1 & x > 1 \\
          x & x \in [0,1] \\
          0 & x < 0
        \end{cases}
      \]
    First let us deal with the random variable $m = \min{\{X_1,...,X_n\}}$:
      \begin{align*}
        F_m(x) &= P(m \leq x) = 1 - P(X_i > x)^n \\
        &= 1 - (1-P(X_i \leq x))^n = 1 - (1 - F_{X_i}(x))^n \\
      \end{align*}
        \[
          \therefore F_m(x) =
          \begin{cases}
            1 & x > 1 \\
            1-(1-x)^n & x \in [0,1] \\
            0 & x < 0
          \end{cases}
        \]
        \[
          \therefore f_m(x) = F_m'(x) =
          \begin{cases}
            n(1-x)^{n-1} & x \in [0,1] \\
            0 & otherwise
          \end{cases}
        \]
      \begin{equation*}
        \therefore E[m] = \int_0^1 xn(1-x)^{n-1}dx = \frac{1}{n+1}
      \end{equation*}

      Now let us deal with the random variable $M = \max{\{X_1,...,X_n\}}$:
        \begin{align*}
          F_M(x) &= P(M \leq x) = P(X_i \leq x)^n = F_{X_i}(x))^n
        \end{align*}
          \[
            \therefore F_M(x) =
            \begin{cases}
              1 & x > 1 \\
              x^n & x \in [0,1] \\
              0 & x < 0
            \end{cases}
          \]
          \[
            \therefore f_M(x) = F_M'(x) =
            \begin{cases}
              nx^{n-1} & x \in [0,1] \\
              0 & otherwise
            \end{cases}
          \]
        \begin{equation*}
          \therefore E[M] = \int_0^1 xnx^{n-1}dx = \frac{n}{n+1}
        \end{equation*}
  \item
    \begin{enumerate}
      \item
        Letting $S_n = \sum_{i=1}^{1000}X_i$, where $X_i$ is the probability of a good state in period $i$. Since the $X_i$'s are independent Bernoulli distributed random variables, $E[X_i] = \mu = 0.5$ and $\sigma^2 = 0.25$. We want to estimate $P(S_n \leq 510) - P(S_n \leq 490)$. Thus, using the Central limit theorem, we have that:
          \begin{align*}
            P(S_n \leq 510) & \approx \Upphi(\frac{510 - n\mu}{\sigma\sqrt{n}}) \\
            &= \Upphi(\frac{2}{\sqrt{10}}) \\
            P(S_n \leq 490) & \approx \Upphi(\frac{490 - n\mu}{\sigma\sqrt{n}}) \\
            &= \Upphi(-\frac{2}{\sqrt{10}}) \\
            \therefore P(S_n \leq 510) - P(S_n \leq 490) & \approx 0.47
          \end{align*}
        Therefore, there is a $47\%$ chance that the number of good states over 1000 periods differs from 500 by at most $2\%$.
      \item
        We want to find $n$ such that $P(|\frac{S_n}{n} - \mu| \geq 0.005) \leq 0.01$. From the weak law of large numbers, we have that $\frac{\sigma^2}{n\epsilon^2} = \frac{0.25}{0.005n} = 0.01$. Therefore, $n = 5000$.
    \end{enumerate}
  \item
  Since $E[X]$ is finite, we can use Jensen's inequality. Since $f(x) = e^{\theta x}$ is a differentiable convex function, according to Jensen's inequality, $E[e^{\theta X}] = 1 \geq e^{\theta E[X]}$. Since $E[X] < 0$, then $\theta > 0$ or else $1 \leq e^{\theta E[X]}$, which would contradict Jensen's inequality.

\end{enumerate}

\end{document}
