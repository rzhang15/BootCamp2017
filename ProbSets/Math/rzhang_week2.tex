\documentclass[letterpaper,12pt]{article}
\usepackage{array}
\usepackage{threeparttable}
\usepackage{geometry}
\geometry{letterpaper,tmargin=1in,bmargin=1in,lmargin=0.5in,rmargin=0.5in}
\usepackage{fancyhdr,lastpage}
\pagestyle{fancy}
\lhead{}
\chead{}
\rhead{}
\lfoot{\footnotesize\textsl{OSM Lab, Summer 2017, Math PS \#2}}
\cfoot{}
\rfoot{\footnotesize\textsl{Page \thepage\ of \pageref{LastPage}}}
\renewcommand\headrulewidth{0pt}
\renewcommand\footrulewidth{0pt}
\usepackage[format=hang,font=normalsize,labelfont=bf]{caption}
\usepackage{amsmath}
\usepackage{amssymb}
\usepackage{amsthm}
\usepackage{natbib}
\usepackage{upgreek}
\usepackage{setspace}
\usepackage{float,color}
\usepackage[pdftex]{graphicx}
\usepackage{hyperref}
\usepackage{mathrsfs}
\usepackage{dsfont}
\hypersetup{colorlinks,linkcolor=red,urlcolor=blue,citecolor=red}
\theoremstyle{definition}
\newtheorem{theorem}{Theorem}
\newtheorem{acknowledgement}[theorem]{Acknowledgement}
\newtheorem{algorithm}[theorem]{Algorithm}
\newtheorem{axiom}[theorem]{Axiom}
\newtheorem{case}[theorem]{Case}
\newtheorem{claim}[theorem]{Claim}
\newtheorem{conclusion}[theorem]{Conclusion}
\newtheorem{condition}[theorem]{Condition}
\newtheorem{conjecture}[theorem]{Conjecture}
\newtheorem{corollary}[theorem]{Corollary}
\newtheorem{criterion}[theorem]{Criterion}
\newtheorem{definition}[theorem]{Definition}
\newtheorem{derivation}{Derivation} % Number derivations on their own
\newtheorem{example}[theorem]{Example}
\newtheorem{exercise}[theorem]{Exercise}
\newtheorem{lemma}[theorem]{Lemma}
\newtheorem{notation}[theorem]{Notation}
\newtheorem{problem}[theorem]{Problem}
\newtheorem{proposition}{Proposition} % Number propositions on their own
\newtheorem{remark}[theorem]{Remark}
\newtheorem{solution}[theorem]{Solution}
\newtheorem{summary}[theorem]{Summary}
%\numberwithin{equation}{section}
\bibliographystyle{aer}
\newcommand\ve{\varepsilon}
\newcommand\boldline{\arrayrulewidth{1pt}\hline}

\begin{document}

\title{Math Problem Set 2 \\
Open Source Macroeconomics Laboratory Boot Camp}
\author{Ruby Zhang}
\maketitle

\begin{enumerate}
  \item[3.1]
    \begin{enumerate}
      \item[(i)]
        \begin{align*}
          \frac{1}{4}(\| x+y \|^2 - \|x-y\|^2) &= \frac{1}{4}(\langle x+y,x+y\rangle - \langle x-y,x-y \rangle) \\
          &= \frac{1}{4}(\langle x+y,x \rangle + \langle x+y, y \rangle - (\langle x-y,x \rangle - \langle x-y, y \rangle)) \\
          &= \frac{1}{4}(\langle x,x \rangle + \langle y,x \rangle+ \langle x,y \rangle + \langle y,y \rangle - (\langle x,x \rangle - \langle y,x \rangle - \langle x,y \rangle + \langle y,y \rangle)) \\
          &=\frac{1}{4}(4\langle x,y \rangle ) \\
          &= \langle x,y \rangle
        \end{align*}
      \item[(ii)]
        \begin{align*}
          \frac{1}{2}(\| x+y \|^2 + \|x-y\|^2) &= \frac{1}{4}(\langle x+y,x+y\rangle - \langle x-y,x-y \rangle) \\
          &= \frac{1}{2}(\langle x+y,x \rangle + \langle x+y, y \rangle + \langle x-y,x \rangle - \langle x-y, y \rangle) \\
          &= \frac{1}{2}(\langle x,x \rangle + \langle y,x \rangle+ \langle x,y \rangle + \langle y,y \rangle + \langle x,x \rangle - \langle y,x \rangle - \langle x,y \rangle + \langle y,y \rangle) \\
          &= \frac{1}{2}(2\langle x,x \rangle + 2\langle y,y \rangle) \\
          &= \langle x,x \rangle + \langle y,y \rangle \\
          &= \|x\|^2 + \|y\|^2
        \end{align*}
    \end{enumerate}

    \item[3.2]
    \begin{align*}
      & \frac{1}{4}(\| x+y \|^2 - \|x-y\|^2 + i\| x-iy \|^2 - i\|x+iy\|^2) \\
      &= \frac{1}{4}(\langle x+y,x+y\rangle - \langle x-y,x-y \rangle +i\langle x-iy,x-iy\rangle - i\langle x+iy,x+iy \rangle ) \\
      &= \frac{1}{4}(2\langle x,y \rangle + 2\langle y,x \rangle - i(2\langle x,iy \rangle + 2\langle iy,x \rangle)) \\
      &=  \frac{1}{4}(2\langle x,y \rangle + 2\langle y,x \rangle- i(2i\langle x,y \rangle - 2i\langle y,x \rangle)) \\
      &= \frac{1}{4}(2\langle x,y \rangle + 2\langle y,x \rangle + 2\langle x,y \rangle - 2\langle y,x \rangle) \\
      &= \frac{1}{4}(4\langle x,y \rangle) \\
      &= \langle x,y \rangle
    \end{align*}

    \item[3.3]
      \begin{enumerate}
        \item[(i)]
          \begin{align*}
            \cos \theta &= \frac{\langle x,x^5 \rangle}{\|x\| \|x^5\|} = \frac{\int_0^1 x^6 dx}{\sqrt{\int_0^1 x^2 dx \int_0^1 x^{10} dx}} = \frac{\frac{1}{7}}{\sqrt{\frac{1}{3}\frac{1}{11}}} \\
            \therefore \theta &=\arccos{\frac{\sqrt{33}}{7}} = 0.608
          \end{align*}
        \item[(ii)]
          \begin{align*}
            \cos \theta &= \frac{\langle x^2,x^4 \rangle}{\|x^2\| \|x^4\|} = \frac{\int_0^1 x^6 dx}{\sqrt{\int_0^1 x^4 dx \int_0^1 x^{8} dx}} = \frac{\frac{1}{7}}{\sqrt{\frac{1}{5}\frac{1}{9}}} \\
            \therefore \theta &=\arccos{\frac{\sqrt{45}}{7}} = 0.29
          \end{align*}
      \end{enumerate}

    \item[3.8]
      \begin{enumerate}
        \item[(i)]
        We must prove that $S = \{\cos(t), \sin(t), \cos(2t), \sin(2t)\}$ is an orthnormal set. First let us prove that all pairs of the basis are orthogonal:
          \begin{align*}
            \langle \cos(t), \sin(t) \rangle &= \frac{1}{\pi}\int_{-\pi}^\pi \sin(t)\cos(t)dt = \frac{1}{\pi}\int_{-\pi}^\pi \sin(t)d\sin(t) \\
            &= \frac{1}{2\pi} \sin^2(t)]_{-\pi}^\pi = 0 \\
            \langle \cos(t), \cos(2t) \rangle &=\frac{1}{\pi}\int_{-\pi}^\pi \cos(2t)\cos(t)dt =\frac{1}{\pi}\int_{-\pi}^\pi \frac{\cos(2t+t) + \cos(2t-t)}{2}dt \\
            &=\frac{1}{2\pi} \int_{-\pi}^\pi \cos(3t) + \cos(t) dt = \frac{1}{2\pi} [\frac{1}{3}\sin(3t) + \sin(t)]_{-\pi}^\pi = 0 \\
            \langle \cos(t), \sin(2t) \rangle &= \frac{1}{\pi}\int_{-\pi}^\pi \sin(2t)\cos(t)dt = \frac{2}{\pi}\int_{-\pi}^\pi \sin(t)\cos^2(t)dt \\
            &= \frac{-2}{\pi}\int_{-\pi}^\pi \cos^2(t)d\cos(t) = \frac{-2}{3\pi} [\cos^3(t)]_{-\pi}^\pi = 0 \\
            \langle \cos(2t), \sin(t) \rangle &= \frac{1}{\pi}\int_{-\pi}^\pi \sin(t)\cos(2t)dt = \frac{1}{\pi}\int_{-\pi}^\pi \frac{\sin(2t+t) - \sin(2t-t)}{2}dt \\
            &=\frac{1}{2\pi} \int_{-\pi}^\pi \sin(3t) - \sin(t) dt = \frac{1}{2\pi} [-\frac{1}{3}\cos(3t) + \cos(t)]_{-\pi}^\pi = 0 \\
            \langle \sin(2t), \sin(t) \rangle &= \frac{1}{\pi}\int_{-\pi}^\pi \sin(t)\sin(2t)dt = \frac{2}{\pi}\int_{-\pi}^\pi \sin^2(t)\cos(t)dt \\
            &= \frac{2}{\pi}\int_{-\pi}^\pi \sin^2(t)d\sin(t) = \frac{2}{3\pi}[\sin^3(t)]_{-\pi}^\pi = 0 \\
            \langle \sin(2t), \cos(2t) \rangle &=  \frac{1}{\pi}\int_{-\pi}^\pi \cos(2t)\sin(2t)dt = -\frac{1}{2\pi}\int_{-\pi}^\pi \cos(2t)d\cos(2t) \\
            &= -\frac{1}{4\pi} [\cos^2(2t)]_{-\pi}^\pi = 0
          \end{align*}
          Now we will prove that the norm of each basis element is equal to 1:
          \begin{align*}
              \langle \cos(t), \cos(t) \rangle &= \frac{1}{\pi}\int_{-\pi}^\pi \cos^2(t)dt = \frac{1}{\pi}\int_{-\pi}^\pi \frac{1 + cos(2t)}{2}dt \\
              &= \frac{1}{2\pi} [t + \frac{1}{2}\sin(2t)]_{-\pi}^\pi = \frac{1}{2\pi}2\pi = 1 \\
              \langle \sin(t), \sin(t) \rangle &= \frac{1}{\pi}\int_{-\pi}^\pi \sin^2(t)dt = \frac{1}{\pi}\int_{-\pi}^\pi \frac{1 - cos(2t)}{2}dt \\
              &= \frac{1}{2\pi} [t - \frac{1}{2}\sin(2t)]_{-\pi}^\pi = \frac{1}{2\pi}2\pi = 1 \\
              \langle \cos(2t), \cos(2t) \rangle &= \frac{1}{\pi}\int_{-\pi}^\pi \cos^2(2t)dt = \frac{1}{\pi}\int_{-\pi}^\pi \frac{1 + cos(4t)}{2}dt \\
              &= \frac{1}{2\pi} [t + \frac{1}{4}\sin(4t)]_{-\pi}^\pi = \frac{1}{2\pi}2\pi = 1 \\
              \langle \sin(2t), \sin(2t) \rangle &= \frac{1}{\pi}\int_{-\pi}^\pi \sin^2(2t)dt = \frac{1}{\pi}\int_{-\pi}^\pi \frac{1 - cos(4t)}{2}dt \\
              &= \frac{1}{2\pi} [t - \frac{1}{4}\sin(4t)]_{-\pi}^\pi = \frac{1}{2\pi}2\pi = 1 \\
          \end{align*}
        \item[(ii)]
          \begin{align*}
            \|t\| &= \sqrt{\langle t,t \rangle} = \sqrt{\frac{1}{\pi}\int_{-\pi}^\pi t^2 dt} = \sqrt{\frac{2\pi^2}{3}} = \sqrt{\frac{2}{3}}\pi
          \end{align*}
        \item[(iii)]
          \begin{align*}
            \text{proj}_X(\cos(3t)) &= \langle \cos(3t), \cos(t) \rangle \cos(t) + \langle \cos(3t), \sin(t) \rangle \sin(t) \\
            &+ \langle \cos(3t), \cos(2t) \rangle \cos(2t) + \langle \cos(3t), \sin(2t) \rangle \sin(2t) \\
            &= \frac{1}{\pi}(\cos(t)\int_{-\pi}^\pi \cos(3t)\cos(t) dt + \sin(t)\int_{-\pi}^\pi \cos(3t)\sin(t) dt  \\
            &+ \cos(2t)\int_{-\pi}^\pi \cos(3t)\cos(2t) dt + \sin(2t)\int_{-\pi}^\pi \cos(3t)\sin(2t)dt) \\
            &= 0
          \end{align*}
        \item[(iv)]
          \begin{align*}
            \text{proj}_X(t) &= \langle t, \cos(t) \rangle \cos(t) + \langle t, \sin(t) \rangle \sin(t) \\
            &+ \langle t, \cos(2t) \rangle \cos(2t) + \langle t, \sin(2t) \rangle \sin(2t) \\
            &= \frac{1}{\pi}(\cos(t)\int_{-\pi}^\pi t\cos(t) dt + \sin(t)\int_{-\pi}^\pi t\sin(t) dt  \\
            &+ \cos(2t)\int_{-\pi}^\pi t\cos(2t) dt + \sin(2t)\int_{-\pi}^\pi t\sin(2t)dt) \\
            &= \frac{1}{\pi}(2\pi\sin(t) - \pi\sin(2t))\\
            &= 2\sin(t) - \sin(2t)
          \end{align*}
      \end{enumerate}

      \item[3.9]
        Let $L_\theta: \mathbf{R}^2 \rightarrow \mathbf{R}^2$ be the rotation transformation around the origina counterclockwise by angle $\theta$:
          \begin{equation*}
            L_\theta(x,y) = (x\cos\theta - y\sin\theta, x\sin\theta+y\cos\theta)
          \end{equation*}
          Now let us calculate the inner product after the transformation for vectors $\mathbf{u} = (x_1, y_1)$ and $\mathbf{v} = (x_2, y_2)$:
          \begin{align*}
            \langle L_\theta \mathbf{u},L_\theta \mathbf{v} \rangle &= (x_1\cos\theta - y_1\sin\theta)(x_2\cos\theta - y_2\sin\theta)+ (x_1\sin\theta+y_1\cos\theta)(x_2\sin\theta+y_2\cos\theta) \\
            &= x_1x_2 (\cos^2\theta + \sin^2\theta) + y_1y_2(\sin^2\theta + \cos^2\theta)  \\
            &- (x_1y_2 + x_2y_1)\cos\theta\sin\theta + (x_1y_2 + x_2y_1)\cos\theta\sin\theta \\
            &= x_1x_2 + y_1y_2 \\
            &= \langle \mathbf{u}, \mathbf{v} \rangle
          \end{align*}
          Therefore, the rotation transformation is an orthonormal transformation.

        \item[3.10]
          \begin{enumerate}
            \item[(i)]
              $\Longrightarrow$
              \begin{align*}
                \langle Qx, Qy \rangle &= (Qx)^H(Qy) = (x^H Q^H)Qy = x^H (Q^HQ y) = \langle x,Q^HQy \rangle
              \end{align*}
              Since $Q$ is an orthonormal transformation, we know that $ \langle x,Q^HQy \rangle =\langle x,y \rangle = \langle x,Iy \rangle$. Therefore, $Q^HQ = I$.

              $ \Longleftarrow$

                We know that $Q^HQ = I$, thus:
                \begin{align*}
                  \langle Qx, Qy \rangle &= (Qx)^H(Qy) = (x^H Q^H)Qy = x^H (Q^HQ) y = x^Hy = \langle x,y \rangle
                \end{align*}
            \item[(ii)]
            By the orthonormality of matrix $Q$, we have that
              \begin{align*}
                \|Qx\|^2 = \langle Qx, Qx \rangle = \langle x, x \rangle = \|x\|^2
              \end{align*}
            Since the norm is nonnegative, then $\|Qx\| = \|x\|$.
            \item[(iii)]
            Since $Q$ is orthonormal, $QQ^H=I$ so $Q^H=Q^{-1}$. Note that $I=QQ^H=(Q^H)^HQ^H=(Q^{-1})^HQ^{-1}$. By (i), $Q^{-1}$ is also orthnormal.
            \item[(iv)]
            Suppose we have the orthonormal matrix $Q = [a_1,...,a_n]$ where ${a_i}$'s are column vectors. Then the element in the $i$th row and $j$th column of matrix $Q^HQ = I$ is given by the entry $a_i^Ha_j = \delta_{ij}$. Therefore, $a_i^Ha_j = \langle a_i, a_j \rangle$ = 0 for all $i \neq j$ and 0 otherwise, which makes $\{a_i\}$ a collection of orthnormal vectors.
            \item[(v)]
            If $Q$ is orthogonal, then $\det^2(Q) = \det(Q)\det(Q^H) = \det(QQ^H) = \det(I)=1$. Therefore, $\sqrt{\det^2(Q)} = |\det(Q)| = 1$. However, the converse is not true. Take the following matrix:
            \[
              A =
              \begin{bmatrix}
                2 & 1 \\
                1 & 1
              \end{bmatrix}
            \]
            Note that $\det(A) = 2 - 1 = 1$ but the columns are not orthonormal.
            \item[(vi)]
            \begin{equation*}
              (Q_1Q_2)(Q_1Q_2)^H = (Q_1Q_2)(Q_2^HQ_1^H) = Q_1(Q_2Q_2^H)Q_1^H) = Q_1Q_1^H = I
            \end{equation*}
            According to (i), $Q_1Q_2$ is orthnormal.
          \end{enumerate}

        \item[3.11]
        Suppose we have the vectors $\{v_1,...,v_n\}$ where vector $v_n$ is linearly dependent on the vectors $\{v_1,...,v_{n-1}\}$, which are linearly independent. Then the process would output the 0 vector for vector $q_n$ since $p_{n-1} = v_k$ as the projection of the vector onto the space is itself since it is linearly dependent with the basis of the space.

        \item[3.16]
          \begin{enumerate}
            \item[(i)]
            For any diagnal matrix $D$ and QR decomposition, we have that $QR = QIR = QDD^{-1}R = (QD)(D^{-1}R)$. Note that all diagonal matrices (and its inverse) are both orthonormal $D = D^{-1} = D^H$ and upper triangular. Since orthonormal and triangular matrices are closed under multiplication, then $Q'R'$, where $Q' = QD$ and $R' = D^{-1}R$, is another QR decomposition of the same matrix.
            \item[(ii)]
            Suppose there exist two QR decompositions of $A$: $Q_1R_1 = A = Q_2R_2$. Then we have that $B = Q_2^HQ^1 = R_2R_1^{-1}$. Since orthnormal and upper triangular matrices are closed under multiplication and inverses, then $B$ is an orthonormal, upper triangular matrix. Then $B$ must be a diagonal matrix since $B_H = B_{-1}$ which would not hold if there were nonzero, nondiagonal entries as that would produce a lower triangular matrix which violate the closing of upper triangular matrices under the inverse. In addition, since the columns of $B$ must be orthonormal, then all the diagonal entries are $\pm1$. But we know that $R_1, R_2$ have positive diagonal entries so $B = I$. Thus, $I = Q_2^HQ^1 = R_2R_1^{-1}$ so we have that $Q_2 = Q_1$ and $R_1 = R_2$.
          \end{enumerate}

        \item[3.17]
          \begin{align*}
            A^HA\mathbf{x} &= A^H\mathbf{b} \\
            (\widehat{Q}\widehat{R})^H(\widehat{Q}\widehat{R})\mathbf{x} &=   (\widehat{Q}\widehat{R})^H\mathbf{b} \\
            \widehat{R}^H(\widehat{Q}^H\widehat{Q})\widehat{R}\mathbf{x} &=   \widehat{R}^H\widehat{Q}^H\mathbf{b} \\
            \widehat{R}^H\widehat{R}\mathbf{x} &= \widehat{R}^H\widehat{Q}^H\mathbf{b} \\
            \widehat{R}\mathbf{x} &= \widehat{Q}^H\mathbf{b} \\
          \end{align*}

        \item[3.23]
          Note that using the triangle inequality property of the norm, we have that:
          \begin{align*}
            \|y\| &= \|x+(y-x)\| \leq \|x\| + \|y-x\| \\
            \therefore \|y\| - \|x\| & \leq \|y-x\| = \|-1\|\|x-y\| = \|x-y\| \\
            \|x\| &= \|y+(x-y)\| \leq \|y\| + \|x-y\| \\
            \therefore \|x\| - \|y\| & \leq \|x-y\| \\
          \end{align*}
          \begin{equation*}
            \therefore |\|x\| - \|y\|| \leq \|x-y\|
          \end{equation*}

        \item[3.24]
          We must prove that each of the following satisfies positivity (and equality), scale preservation, and triangle inequality:
          \begin{enumerate}
            \item[(i)]
              \begin{enumerate}
                \item[1.]
                Since $|f(t)|$ is a nonnegative function, its integral is also nonnegative. If $\|f\|_{L^1} = \int_a^b|f(t)|dt = 0$, then $f(t)=0$ on $[a,b]$ since $f$ is continuous on $[a,b]$.
                \item[2.]
                $\|\alpha f\|_{L^1} = \int_a^b|\alpha f(t)|dt = |\alpha| \int_a^b|f(t)|dt = |\alpha| \|f\|_{L^1}$
                \item[3.]
                  \begin{align*}
                    \|f+g\|_{L^1} &= \int_a^b|f(t) + g(t)|dt \leq \int_a^b|f(t)| + |g(t)|dt \\
                    &= \int_a^b|f(t)|dt + \int_a^b|g(t)|dt = \|f\|_{L^1}+\|g\|_{L^1}
                  \end{align*}
              \end{enumerate}
            \item[(ii)]
              \begin{enumerate}
                \item[1.]
                  Since $|f(t)|^2$ is a nonnegative function, its integral and the square root of it are also nonnegative. If $\|f\|_{L^2} = (\int_a^b|f(t)|^2dt)^{\frac{1}{2}} = 0$, then $f(t)=0$ on $[a,b]$ since $f$ is continuous on $[a,b]$.
                \item[2.]
                $\|\alpha f\|_{L^2} = (\int_a^b|\alpha f(t)|^2dt)^{\frac{1}{2}} = (|\alpha|^2 \int_a^b|f(t)|^2dt)^{\frac{1}{2}} = |\alpha| \|f\|_{L^2}$
                \item[3.]
                \begin{align*}
                  \|f+g\|_{L^2}^2 &= \int_a^b|f(t) + g(t)|^2 dt) = \int_a^b|f(t)|^2dt + \int_a^b|g(t)|^2dt + 2\int_a^bf(t)g(t)dt \\
                  & \leq \int_a^b|f(t)|^2dt + \int_a^b|g(t)|^2dt + 2\sqrt{\int_a^b|f(t)g(t)|^2dt} \text{ by the Schwarz inequality} \\
                  &= \Bigg(\sqrt{\int_a^b|f(t)|^2dt} + \sqrt{\int_a^b|g(t)|^2dt} \Bigg)^2 \\
                  \therefore \|f+g\|_{L^2} &\leq \sqrt{\int_a^b|f(t)|^2dt} + \sqrt{\int_a^b|g(t)|^2dt} = \|f\|_{L^2} + \|g\|_{L^2}
                \end{align*}
              \end{enumerate}
            \item[(iii)]
              \begin{enumerate}
                \item[1.]
                Since $|f(t)|$ is a nonnegative function, its supremum is also nonnegative. If $\sup_{x\in[a,b]}|f(x)| = 0$, then $f(t)=0$ on $[a,b]$ since any non-zero value of $f(t)$ on $[a,b]$ would contradict $\sup_{x\in[a,b]}|f(x)| = 0$.
                \item[2.]
                $\|\alpha f\|_{L^\infty} = \sup_{x\in[a,b]}|\alpha f(x)| = |\alpha| \sup_{x\in[a,b]}|f(x)| = |\alpha|\|f\|_{L^\infty}$
                \item[3.]
                \begin{align*}
                  \|f+g\|_{L^\infty} &= \sup_{x\in[a,b]}|f(x) + g(x)| \leq \sup_{x\in[a,b]}|f(x)| + |g(x)| \\
                  & \leq \sup_{x\in[a,b]}|f(x)| + \sup_{x\in[a,b]}|g(x)| =\|f\|_{L^\infty} + \|g\|_{L^\infty}
                \end{align*}
              \end{enumerate}
          \end{enumerate}

        \item[3.26]
        First, we will prove that topological equivalence (denote by $\sim$) is an equivalence relation:
          \begin{enumerate}
            \item[1.] Reflexivity:
            We have that $0.5\|x\|_a \leq \|x\|_a \leq 2\|x\|_a$ for all $x \in X$. Thus, $\|x\|_a \sim \|x\|_a$.
            \item[2.] Symmetry:
            Suppose $\|x\|_a \sim \|x\|_b$. Then there exists constants $0 < m \leq M$ such that $m\|x\|_a \leq \|x\|_b \leq M\|x\|_a$. We then have that $\frac{1}{M}\|x\|_b \leq \|x\|_a \leq \frac{1}{m}\|x\|_b$ where $0 < \frac{1}{M} \leq \frac{1}{m}$. Thus, $\|x\|_b \sim \|x\|_a$ as well.
            \item[3.] Transitivity:
            Suppose $\|x\|_a \sim \|x\|_b$ and $\|x\|_b \sim \|x\|_c$. Then there exist constants $0 < m_1 \leq M_1$ and $0 < m_2 \leq M_2$ such that $m_1\|x\|_a \leq \|x\|_b \leq M_1\|x\|_a$ and $m_2\|x\|_b \leq \|x\|_c \leq M_2\|x\|_b$. Then we have that $m_1m_2\|x\|_a \leq \|x\|_c \leq M_1M_2\|x\|_a$ where $0 < m_1m_2 \leq M_1M_2$. Therefore, $\|x\|_a \sim \|x\|_c$.
          \end{enumerate}
          Now we will show that the $p$-norms for $p = 1,2,\infty$ are topologically equivalent by establishing the following inequalities:
          \begin{enumerate}
            \item[(i)]
              First, note that we have:
              \begin{align*}
                \Big(\sum_{i=1}^n |x_i|\Big)^2 &= \sum_{i=1}^n |x_i|^2 + \sum_{i\neq j}^n |x_i||x_j| \\
                & \geq \sum_{i=1}^n |x_i|^2 \text{  since all terms are nonnegative} \\
                \sum_{i=1}^n |x_i| & \geq \sqrt{\sum_{i=1}^n |x_i|^2} \\
                \therefore  \|x\|_2 & \leq \|x\|_1
              \end{align*}
              From the Cauchy-Schwarz inequality, we have that $\Big|\sum_{i=1}^n x_iy_i\Big|^2 \leq \sum_{j=1}^n |x_j|^2 \sum_{k=1}^n |y_j|^2$. Letting $y_i = 1$ for all $1 \leq i \leq n$, we have that
              \begin{align*}
                \Big(\sum_{i=1}^n 1|x_i|\Big)^2 & \leq \sum_{i=1}^n |x_i|^2 \sum_{i=1}^n 1 = n \sum_{i=1}^n |x_i|^2 \\
                \sum_{i=1}^n 1|x_i| & \leq \sqrt{n \sum_{i=1}^n |x_i|^2} \\
                \therefore \|x\|_1 & \leq \sqrt{n}\|x\|_2
              \end{align*}
              Therefore, $\|x\|_1 \sim \|x\|_2$ since $\|x\|_2 \leq \|x\|_1 \leq \|x\|_2$.
            \item[(ii)]
            Note that $\|x\|_\infty = \max_i |x_i|$. Suppose $\max_i |x_i| = |x_j|$. Therefore, $|x_j|^2 \leq |x_j|^2+\sum_{i \neq j}^n |x_i|^2 \leq \sum_{i=1}^n |x_j|^2 = n|x_j|^2$. Then we have that $ (\max_i |x_i|)^2 \leq \sum_{ =1}^n |x_i|^2 \leq n(\max_i |x_i|)^2$. Therefore, $\|x\|_\infty \leq \|x\|_2 \leq \sqrt{n}\|x\|_\infty$ so $\|x\|_2 \sim \|x\|_\infty$.
          \end{enumerate}
          Since $\sim$ is an equivalence relation, transitivity holds so $\|x\|_1 \sim \|x\|_\infty$.

        \item[3.28]
          \begin{enumerate}
            \item[(i)]
            From question 3.26(i), we have that $\frac{1}{\|x\|_2} \leq \frac{\sqrt{n}}{\|x\|_1}$ and $\frac{1}{\|x\|_1} \leq \frac{1}{\|x\|_2}$ for all $x$. Also using the direct inequality proved in 3.26(i), we have that:
            \begin{align*}
              \frac{1}{\sqrt{n}}\|A\|_2 &= \frac{1}{\sqrt{n}} \sup_{x\neq0} \frac{\|Ax\|_2}{\|x\|_2} \leq \frac{1}{\sqrt{n}} \sup_{x\neq0} \sqrt{n}\frac{\|Ax\|_1}{\|x\|_1} \\
              &= \|A\|_1 = \sup_{x\neq0} \frac{\|Ax\|_1}{\|x\|_1} \leq \sup_{x\neq0} \sqrt{n}\frac{\|Ax\|_2}{\|x\|_2} \\
              &= \sqrt{n}\|A\|_2 \\
              \therefore \frac{1}{\sqrt{n}}\|A\|_2 & \leq \|A\|_1 \leq\sqrt{n}\|A\|_2
            \end{align*}
            \item[(ii)]
            Similarly, using 3.26(ii), we have that $\frac{1}{\|x\|_\infty} \leq \frac{\sqrt{n}}{\|x\|_2}$ and $\frac{1}{\|x\|_2} \leq \frac{1}{\|x\|_\infty}$ for all $x$. Also using the direct inequality proved in 3.26(ii), we have that:
            \begin{align*}
              \frac{1}{\sqrt{n}}\|A\|_\infty &= \frac{1}{\sqrt{n}} \sup_{x\neq0} \frac{\|Ax\|_\infty}{\|x\|_\infty} \leq \frac{1}{\sqrt{n}} \sup_{x\neq0} \sqrt{n}\frac{\|Ax\|_2}{\|x\|_2} \\
              &= \|A\|_2 = \sup_{x\neq0} \frac{\|Ax\|_2}{\|x\|_2} \leq \sup_{x\neq0} \sqrt{n}\frac{\|Ax\|_\infty}{\|x\|_\infty} \\
              &= \sqrt{n}\|A\|_\infty \\
              \therefore \frac{1}{\sqrt{n}}\|A\|_\infty & \leq \|A\|_2 \leq\sqrt{n}\|A\|_\infty
            \end{align*}
          \end{enumerate}

        \item[3.29]
        From exercise 3.10(ii), we proved that $\|Q\mathbf{x}\|_2 = \|\mathbf{x}\|_2$ for all $\mathbf{x} \in \mathbb{F}^n$ and orthnormal matrix $Q$. Therefore, $\|Q\| = \sup_{x\neq0}\frac{\|Q\mathbf{x}\|}{\|\mathbf{x}\|} = 1$.
        The induced norm of $R_{\mathbf{x}}$ is given by: $\|R_{\mathbf{x}}\|_2 = \sup_{A\neq0} \frac{\|R_{\mathbf{x}}A\|_2}{\|A\|_2} = \sup_{A\neq0}\frac{\|A\mathbf{x}\|_2}{\|A\|_2}$. Since $\|A\|_2 = \sup_{\mathbf{y}\neq0} \frac{\|A\mathbf{y}\|_2}{\|\mathbf{y}\|_2} \geq \frac{\|A\mathbf{x}\|_2}{\|\mathbf{x}\|_2} $, then $\|\mathbf{x}\|_2 \geq \frac{\|A\mathbf{x}\|_2}{\|A\|_2}$ for all matrices $A$ so $\|\mathbf{x}\|_2 \geq \|R_{\mathbf{x}}\|_2$. Note that equality is possible when $A$ is orthonormal since $\|A\mathbf{x}\|_2 = \|\mathbf{x}\|_2 = \|A\|_2 \|\mathbf{x}\|_2$, thus $\frac{\|A\mathbf{x}\|_2}{\|A\|_2} = \|\mathbf{x}\|_2$. Therefore, $\|R_{\mathbf{x}}\|_2 = \sup_{A\neq0} \frac{\|R_{\mathbf{x}}A\|_2}{\|A\|_2} = \sup_{A\neq0}\frac{\|A\mathbf{x}\|_2}{\|A\|_2} = \|\mathbf{x}\|_2$.

        \item[3.30]
        We will first show that $\|\dot\|_S$ satisfies the properties of a norm, then that it satisfies the submultiplicative property of the matrix norm.
          \begin{enumerate}
            \item[1.] Positivity is obviously satisfied since $\|\cdot\|$ is a matrix norm.
            \item[2.] Scale preservation: $\|\alpha A\|_S = \|S(\alpha A)S^{-1}\| = |\alpha| \|SAS^{-1}\| = |\alpha|\|A\|_S$.
            \item[3.] Triangle inequality: $\|A + B\|_S = \|S(A+B)S^{-1}\| = \|SAS^{-1} + SBS^{-1}\| \leq \|A\|_S + \|B\|_S$ since matrix multiplication obeys distributive properties from both left and right.
          \end{enumerate}
        Since $\|\cdot\|$ is a matrix norm with the submultiplicative property, we have that:
          \begin{equation*}
            \|AB\|_S = \|S(AB)S^{-1}\| = \|(SAS^{-1})(SBS^{-1})\| \leq \|SAS^{-1}\|\|SBS^{-1}\| =\|A\|_S\|B\|_S
          \end{equation*}

        \item[3.37]
        Let $q = 180x^2-168x+24$. Then for any $p = ax^2 + bx +c$, we have that
        \begin{align*}
          \langle q, p \rangle &= \int_0^1 qp dx = \int_0^1 (180x^2-168x+24)(ax^2 + bx +c) dx \\
          &= \int_0^1 180ax^4 + 180bx^3 + 180cx^2 - 168ax^3 - 168bx^2 - 168cx + 24ax^2 + 24bx + 24c \\
          &= \frac{180a}{5} + \frac{180b}{4} + \frac{180c}{3} - \frac{168a}{4} - \frac{168b}{3} - \frac{168c}{2} + \frac{24a}{3} + \frac{24b}{2} + 24c \\
          &= (36 - 42 + 8)a + (45 - 56 + 12)b + (60-84+24)c \\
          &= 2a+b = L[p]
        \end{align*}

        \item[3.38]
        Letting the basis be $[1,x,x^2]$, then the coordinates of the basis are $[1,0,0], [0,1,0], [0,0,1]$. Thus $p(x) = a + bx + cx^2$ can be represented as $[a, b, c]$. Then the differentiation matrix is the following:
        \[
          D[p](x) =
          \begin{bmatrix}
            0 & 1 & 0 \\
            0 & 0 & 2 \\
            0 & 0 & 0
          \end{bmatrix}
          \begin{bmatrix}
            a \\
            b \\
            c
          \end{bmatrix}
          =
          \begin{bmatrix}
            b \\
            2c \\
            0
          \end{bmatrix}
          = b + 2cx = p'(x)
        \]
        The adjoint of $D$ is the map $D^*$ such that (using integration by parts):
        \begin{align*}
           \langle f, D^*g \rangle &= \langle Df, g \rangle \\
           \int_0^1 f(x)D^*[g](x) dx &= \int_0^1 D[f](x)g(x)dx \\
           &= \Bigg[f(x)g(x)\Bigg]_0^1 - \int_0^1 f(x)g'(x)dx \\
        \end{align*}
        Restricting to polynomials with $f(0) = f(1)$, then we have that $D^* = -D$, which gives us the following adjoint matrix:
        \[
          D^* =
          \begin{bmatrix}
            0 & -1 & 0 \\
            0 & 0 & -2 \\
            0 & 0 & 0
          \end{bmatrix}
        \]

        \item[3.39]
          \begin{enumerate}
            \item[(i)]
            \begin{align*}
              \langle (S+\alpha T)^*(\mathbf{w}), \mathbf{v} \rangle_V &= \langle \mathbf{w}, (S+\alpha T)(\mathbf{v})\rangle_W \\
              &= \langle \mathbf{w}, S\mathbf{v}+\alpha T\mathbf{v}\rangle_W \\
              &=  \langle \mathbf{w}, S\mathbf{v} \rangle_W + \langle \mathbf{w}, \alpha T\mathbf{v}\rangle_W \\
              &= \langle S^*\mathbf{w}, \mathbf{v} \rangle_V + \langle ( T^*(\mathbf{w}), \alpha\mathbf{v}\rangle_V \\
              &= \langle S^*\mathbf{w}, \mathbf{v} \rangle_V + \langle ( \bar{\alpha}T^*(\mathbf{w}), \mathbf{v}\rangle_V \\
              &= \langle (S^* + \bar{\alpha}T^*)(\mathbf{w}),\mathbf{v} \rangle_V \\
              \therefore (S+\alpha T)^* &= S^* + (\alpha T)^* = S^* + \bar{\alpha}T^*
            \end{align*}
            \item[(ii)]
              \begin{equation*}
                \langle (S^*)^*(\mathbf{w}), \mathbf{v} \rangle_V =  \langle \mathbf{w}, S^*\mathbf{v}\rangle_W  = \langle S(\mathbf{w}), \mathbf{v} \rangle_V
              \end{equation*}
              Therefore, $(S^*)^* = S$.
            \item[(iii)]
              \begin{align*}
                \langle (ST)^*(\mathbf{w}), \mathbf{v} \rangle_V &= \langle \mathbf{w}, (ST)(\mathbf{v})\rangle_W \\
                &= \langle \mathbf{w}, S(T\mathbf{v})\rangle_W \\
                &= \langle S^*\mathbf{w}, T\mathbf{v}\rangle_V \\
                &= \langle T^*S^*\mathbf{w}, \mathbf{v}\rangle_W \\
                \therefore (ST)^* &= T^*S^*
              \end{align*}
            \item[(iv)]
              Since the identity matrix is its own adjoint, we have that
              \begin{equation*}
                (T^*)(T^*)^{-1} = I = I^* \Longrightarrow (T^*)(T^*)^{-1} = (T^{-1}T)^* = T^*(T^{-1})^* \Longrightarrow (T^*)^{-1} = (T^{-1})^*
              \end{equation*}
          \end{enumerate}

        \item[3.40]
          \begin{enumerate}
            \item[(i)]
              For matrices $B, C$ and linear operator (also a matrix) $A$, we have that:
              \begin{equation*}
                \langle A^*B, C \rangle = \langle B, AC \rangle = \text{tr}(B^HAC) = \text{tr}((A^HB)^HC) = \langle A^HB, C \rangle
              \end{equation*}
              Therefore, $A^* = A^H$.
            \item[(ii)]
            \begin{equation*}
              \langle A_2A_1^*, A_3 \rangle = \langle A_2A_1^H, A_3 \rangle = \text{tr}(A_1A_2^HA_3) = \text{tr}(A_1(A_2^HA_3)) = \text{tr}(A_2^HA_3A_1) = \langle A_2, A_3A_1 \rangle
            \end{equation*}
            \item[(iii)]
            Using the fact that trace is a linear mapping (i.e. tr($A+B$) = tr($A$) + tr($B$)), we have that
              \begin{align*}
                \langle T_A^* X, Y \rangle &= \langle X, T_A Y \rangle = \langle X, AY - YA \rangle = \text{tr}(X^HAY - X^HYA) \\
                &= \text{tr}(X^HAY) - \text{tr}(X^HYA) = \text{tr}(X^HAY) - \text{tr}(AX^HY) \\
                &= \langle A^HX, Y \rangle - \langle XA^H, Y \rangle = \langle A^*X - XA^*, Y \rangle \\
                &= \langle T_{A^*}X, Y \rangle \\
                \therefore T_A^* &= T_{A^*}
              \end{align*}
          \end{enumerate}

        \item[3.44]
        From 3.40(i) and the fundamental subspaces theorem (3.8.9), we know that $\mathscr{R}(A)^{\perp} = \mathscr{N}(A^H)$. If $A\mathbf{x} = \mathbf{b}$ has a solution $\mathbf{x} \in \mathbb{F}^n$, then $\mathbf{b} \in \mathscr{R}(A)$, in which case by definition $\langle \mathbf{b}, \mathbf{y} \rangle = 0$ for all $\mathbf{y} \in \mathscr{R}(A)^{\perp} = \mathscr{N}(A^H)$. On the other hand, if there exists $\mathbf{y} \in \mathscr{N}(A^H) = \mathscr{R}(A)^{\perp}$ such that $\langle \mathbf{y}, \mathbf{b} \rangle \neq 0$, then by definition $\mathbf{b} \notin \mathscr{R}(A)$ since $\mathbf{b} \neq \mathbf{0}$ so there does not exist a solution to $A\mathbf{x} = \mathbf{b}$.

        \item[3.45]
        From exercise 1.18, we know that:
        \begin{align*}
          \text{Sym}_n(\mathbb{R}) &= \{A \in M_n(\mathbb{R}) | A^T = A\} \\
          \text{Skew}_n(\mathbb{R}) &= \{A \in M_n(\mathbb{R}) | A^T = -A\}
        \end{align*}
        Note that $A^H = A^T$ and $\langle A, B \rangle = \langle B, A \rangle$ for all $A, B \in M_n(\mathbb{R})$ since the conjugate of a real number is itself. Then for all $A \in \text{Sym}_n(\mathbb{R})$ and $B \in \text{Skew}_n(\mathbb{R})$:
          \begin{align*}
            \langle B, A \rangle &= \text{tr}(B^TA) = -\text{tr}(BA) = -\text{tr}(BA^T) = -\text{tr}(A^TB) = - \langle A, B \rangle = - \langle B, A \rangle \\
            \therefore \langle B, A \rangle &= 0
          \end{align*}
          Then by definition, $\text{Skew}_n(\mathbb{R}) \subset \text{Sym}_n(\mathbb{R})^\perp$.

          Now we will show the other way. Let $E_{ij} = e_ie_j^T$, (the zero matrix except for 1 in position (i,j)). Let $A = E_ij + Eji \in \text{Sym}_n(\mathbb{R})$. If $C \in \text{Sym}_n(\mathbb{R})^\perp$, then $\langle C, E_{ij} \rangle + \langle C, E_{ji} \rangle = 0 \Longrightarrow [C]_{ij} + [C]_{ji} = 0$ for all $i, j$. Therefore, $C^T = -C$ so $C \in \text{Skew}_n(\mathbb{R})$.

          Therefore,  $\text{Skew}_n(\mathbb{R}) = \text{Sym}_n(\mathbb{R})^\perp$.

        \item[3.46]
          \begin{enumerate}
            \item[(i)]
            If $\mathbf{x} \in \mathscr{N}(A^HA)$, then $A^H(A\mathbf{x}) = \mathbf{0}$. Therefore, $A\mathbf{x} \in \mathscr{N}(A^H)$ and obviously $\mathscr{R}(A)$ by definition.
            \item[(ii)]
            If $\mathbf{x} \in \mathscr{N}(A)$, then $A^H(A\mathbf{x}) = A^H\mathbf{0} = \mathbf{0}$ so $\mathbf{x} \in \mathscr{N}(A^HA)$.
            If $\mathbf{x} \in \mathscr{N}(A)$, then by previous part and the fundamental subspaces theorem, $A\mathbf{x} \in \mathscr{N}(A^H) = \mathscr{R}(A)^\perp$ and $\mathscr{R}(A)$. Thus, $\langle A\mathbf{x}, A\mathbf{x} \rangle = \mathbf{0}$ so $A\mathbf{x} = 0 \Longrightarrow \mathbf{x} \in \mathscr{N}(A)$.
            \item[(iii)]
            By the rank nullity theorem, $\text{rank}(A) = n - \text{dim}(\mathscr{N}(A)) = n - \text{dim}(\mathscr{N}(A^HA)) = \text{rank}(A^HA)$.
            \item[(iv)]
            If $A$ has linearly independent columns, then it is injective and has rank $n$. Thus, $ n = \text{rank}(A) = \text{rank}(A^HA)$ so $A^HA$ is an $n \times n$ matrix with full rank of $n$, making it nonsingular and bijective.
          \end{enumerate}

        \item[3.47]
          \begin{enumerate}
            \item[(i)]
              \begin{equation*}
                P^2 = (A(A^HA)^{-1}A^H)(A(A^HA)^{-1}A^H) = A(A^HA)^{-1}(A^HA)(A^HA)^{-1}A^H = A(A^HA)^{-1}A^H = P
              \end{equation*}
            \item[(ii)]
            Using the fact that $(B^H)^{-1} = (B^{-1})^H$, we have that
            \begin{align*}
              P^H &= (A(A^HA)^{-1}A^H)^H = (A^H)^H((A^HA)^{-1})^HA^H \\
              &= A((A^HA)^H)^{-1}A^H = A(A^HA)^{-1}A^H\\
              &=P
            \end{align*}
            \item[(iii)]
            Let $\mathbf{x} \in \mathscr{N}(P)$. Then we have that $P\mathbf{x} = A((A^HA)^{-1}A^H\mathbf{x}) = 0 \Longrightarrow (A^HA)^{-1}A^H\mathbf{x} \in \mathscr{N}(A)$. Since rank($A$) = $n$, then dim($\mathscr{N}(A)$) = 0 so $(A^HA)^{-1}(A^H\mathbf{x}) = 0$. Similarly $A^H\mathbf{x} \in \mathscr{N}((A^HA)^{-1})$. From question 3.46(iv), we know that $A^HA$ is nonsingular so rank($(A^HA)^{-1}$) = $n$ and dim($\mathscr{N}((A^HA)^{-1})$) = 0 $\Longrightarrow A^H\mathbf{x} = 0$. Similarly, $\mathbf{x} \in \mathscr{N}(A^H)$ and dim($\mathscr{N}(A^H)$) = dim($\mathscr{R}(A)^\perp$) = 0 so $\mathbf{x} = 0$. Thus, dim($\mathscr{N}(P)$) = 0 and rank($P$) = $n$ by rank nullity theorem.
          \end{enumerate}

        \item[3.48]
          \begin{enumerate}
            \item[(i)]
            For any $A, B \in M_n(\mathbb{R})$ and $\alpha \in \mathbb{R}$, we have that
              \begin{equation*}
                P(A+\alpha B) = \frac{(A+\alpha B)+(A+\alpha B)^T}{2} = \frac{A+A^T}{2} + \alpha\frac{B+ B^T}{2} = P(A) + \alpha P(B)
              \end{equation*}
            which follows from the fact that the transpose is linear since the $(i,j)$ entry of $A^T + B^T$ is the sum of the $(i,j)$ entries of $A^T$ and $B^T$, which is the sum of the $(j,i)$ entries of $A$ and $B$, which is equal to $(i,j)$ entry of $(A+B)^T$.
            \item[(ii)]
            \begin{align*}
              P^2(A) &= P(\frac{P(A)+P(A)^T}{2}) = \frac{1}{2}P(\frac{A+A^T}{2}+\frac{A^T+A}{2}) \\
              &= \frac{1}{2}P(A+A^T) = \frac{1}{2}\frac{A+A^T + A^T+A}{2} = \frac{A+A^T}{2} = P(A)
            \end{align*}
            \item[(iii)]
            \begin{align*}
              \langle P(A), B \rangle &= \text{tr}((\frac{A+A^T}{2})^TB) = \text{tr}(\frac{AB}{2}) + \text{tr}(\frac{A^TB}{2}) = \text{tr}(\frac{A^TB^T}{2}) + \text{tr}(\frac{A^TB}{2}) \\
              &= \text{tr}(A^T\frac{B^T+B}{2}) = \langle A, P(B) \rangle \\
              \therefore P^* &= P
            \end{align*}
            \item[(iv)]
              Note that $P(A) = \frac{A^T+A}{2} = 0$ if and only if $A^T = -A$, in other words if $A \in \text{Skew}_n(\mathbb{R})$.
            \item[(v)]
              If $B \in \mathscr{R}(P)$, then there exists matrix $A$ such that $B = \frac{A + A^T}{2} \Longrightarrow B^T = (\frac{A + A^T}{2})^T = \frac{A + A^T}{2} = B$. Therefore, $B \in \text{Sym}_n(\mathbb{R})$.
              If $B \in \text{Sym}_n(\mathbb{R})$, then $B = B^T$ so $P(B) = \frac{B + B^T}{2} = \frac{2B}{2} = B$ so $B \in \mathscr{R}(P)$.
            \item[(vi)]
              \begin{align*}
                \|A-P(A)\|_F^2 &= \text{tr}((A-P(A))^T(A-P(A))) = \text{tr}((A^T-P(A)^T)(A-P(A))) \\
                &= \text{tr}((A^T-P(A))(A-P(A))) \\
                &= \text{tr}(A^TA) - \text{tr}(A^TP(A)) - \text{tr}(AP(A)) + \text{tr}((P(A))^2) \\
                &= \text{tr}(A^TA) - 2\frac{\text{tr}(A^TA)}{2} - \frac{\text{tr}(A^TA^T)}{2} - \frac{\text{tr}(AA)}{2} + \text{tr}(\frac{A^2 + 2A^TA + (A^T)^2}{4}) \\
                &= -\text{tr}(A^2) + \frac{\text{tr}(A^TA)}{2} + \frac{\text{tr}(A^2)}{2} = \frac{\text{tr}(A^TA)}{2} - \frac{\text{tr}(A^2)}{2} \\
                \therefore \|A-P(A)\|_F &= \sqrt{\frac{\text{tr}(A^TA) - \text{tr}(A^2)}{2}}
              \end{align*}
          \end{enumerate}

        \item[3.50]
        We want to solve the normal equation $A^HA\mathbf{x} = A^H\mathbf{b}$ with the following $A, \mathbf{x}, \mathbf{b}$:
        \[
          A=
          \begin{bmatrix}
            x_1^2 & -1 \\
            x_2^2 & -1 \\
            \vdots & \vdots \\
            rx_n^2 & -1
          \end{bmatrix}
          ,
          \mathbf{x}=
          \begin{bmatrix}
            r \\
            1
          \end{bmatrix}
          , \text{and }
          \mathbf{b}= -s
          \begin{bmatrix}
            y_1^2 \\
            y_2^2 \\
            \vdots \\
            y_n^2
          \end{bmatrix}
        \]

\end{enumerate}

\end{document}
