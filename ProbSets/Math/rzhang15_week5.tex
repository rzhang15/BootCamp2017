\documentclass[letterpaper,12pt]{article}
\usepackage{array}
\usepackage{threeparttable}
\usepackage{geometry}
\geometry{letterpaper,tmargin=1in,bmargin=1in,lmargin=0.5in,rmargin=0.5in}
\usepackage{fancyhdr,lastpage}
\pagestyle{fancy}
\lhead{}
\chead{}
\rhead{}
\lfoot{\footnotesize\textsl{OSM Lab, Summer 2017, Math PS \#5}}
\cfoot{}
\rfoot{\footnotesize\textsl{Page \thepage\ of \pageref{LastPage}}}
\renewcommand\headrulewidth{0pt}
\renewcommand\footrulewidth{0pt}
\usepackage[format=hang,font=normalsize,labelfont=bf]{caption}
\usepackage{amsmath}
\usepackage{amssymb}
\usepackage{amsthm}
\usepackage{natbib}
\usepackage{upgreek}
\usepackage{setspace}
\usepackage{float,color}
\usepackage[pdftex]{graphicx}
\usepackage{hyperref}
\usepackage{mathrsfs}
\usepackage{dsfont}
\hypersetup{colorlinks,linkcolor=red,urlcolor=blue,citecolor=red}
\theoremstyle{definition}
\newtheorem{theorem}{Theorem}
\newtheorem{acknowledgement}[theorem]{Acknowledgement}
\newtheorem{algorithm}[theorem]{Algorithm}
\newtheorem{axiom}[theorem]{Axiom}
\newtheorem{case}[theorem]{Case}
\newtheorem{claim}[theorem]{Claim}
\newtheorem{conclusion}[theorem]{Conclusion}
\newtheorem{condition}[theorem]{Condition}
\newtheorem{conjecture}[theorem]{Conjecture}
\newtheorem{corollary}[theorem]{Corollary}
\newtheorem{criterion}[theorem]{Criterion}
\newtheorem{definition}[theorem]{Definition}
\newtheorem{derivation}{Derivation} % Number derivations on their own
\newtheorem{example}[theorem]{Example}
\newtheorem{exercise}[theorem]{Exercise}
\newtheorem{lemma}[theorem]{Lemma}
\newtheorem{notation}[theorem]{Notation}
\newtheorem{problem}[theorem]{Problem}
\newtheorem{proposition}{Proposition} % Number propositions on their own
\newtheorem{remark}[theorem]{Remark}
\newtheorem{solution}[theorem]{Solution}
\newtheorem{summary}[theorem]{Summary}
%\numberwithin{equation}{section}
\bibliographystyle{aer}
\newcommand\ve{\varepsilon}
\newcommand\boldline{\arrayrulewidth{1pt}\hline}

\begin{document}

\title{Math Problem Set 5 \\
Open Source Macroeconomics Laboratory Boot Camp}
\author{Ruby Zhang}
\maketitle

\begin{enumerate}
  \item[7.1]
    Take any two points $\mathbf{v},\mathbf{u} \in \text{conv}(S)$. Then $\mathbf{v}$ and $\mathbf{u}$ are convex combinations of elements of S, so let's write them as $\mathbf{v} = \alpha_1\mathbf{x}_1+\cdots+\alpha_k\mathbf{x}_k, \mathbf{v} = \beta_1\mathbf{y}_1+\cdots+\beta_n\mathbf{y}_n$ where $\mathbf{x}_i,\mathbf{y}_j \in S, k,n \in \mathbb{N}, \alpha_i,\beta_j \geq 0, \alpha_1+\cdots+\alpha_k = \beta_1+\cdots+\beta_n = 1$. Then for any $\lambda \in [0,1]$, we have that
    \begin{align*}
      \lambda\mathbf{v}+(1-\lambda)\mathbf{u} &= \lambda(\alpha_1\mathbf{x}_1+\cdots+\alpha_k\mathbf{x}_k) + (1-\lambda)(\beta_1\mathbf{y}_1+\cdots+\beta_n\mathbf{y}_n) \\
      &= \lambda\alpha_1\mathbf{x}_1+\cdots+\lambda\alpha_k\mathbf{x}_k + (1-\lambda)\beta_1\mathbf{y}_1+\cdots+(1-\lambda)\beta_n\mathbf{y}_n)
    \end{align*}
    Note that
    \begin{align*}
      \lambda\alpha_1+\cdots+\lambda\alpha_k + (1-\lambda)\beta_1 + \cdots + (1-\lambda)\beta_n &= \lambda(\alpha_1+\cdots+\alpha_k)+(1-\lambda)(\beta_1+\cdots+\beta_n) \\
      &= \lambda+(1-\lambda) = 1
    \end{align*}
    By definition, the result is also a convex combination of elements of $S$, which means that any convec combination of $\mathbf{v}$ and $\mathbf{u}$ are in conv($S$), thus making conv($S$) a convex set.
  \item[7.2]
    \begin{enumerate}
      \item[(i)]
        Take any two elements $\mathbf{x_1},\mathbf{x_2}$ in the hyperplane $P = \{\mathbf{x}\in V|\langle\mathbf{a},\mathbf{x}\rangle=b\}$. For all $\lambda\in[0,1]$, we have that:
        \begin{align*}
          \langle\mathbf{a},\lambda\mathbf{x_1}+(1-\lambda)\mathbf{x_2}\rangle &= \lambda\langle\mathbf{a},\mathbf{x-1}\rangle + (1-\lambda)\langle\mathbf{a},\mathbf{x_2}\rangle \\
          &= \lambda b + (1-\lambda)b = b
        \end{align*}
        Therefore, any convex combination of two points in $P$ is still in hyperplane $P$ so hyperplanes are convex.
      \item[(ii)]
      Take any two elements $\mathbf{x_1},\mathbf{x_2}$ in the half-space $H = \{\mathbf{x}\in V|\langle\mathbf{a},\mathbf{x}\rangle\leq b\}$. For all $\lambda\in[0,1]$, we have that:
      \begin{align*}
        \langle\mathbf{a},\lambda\mathbf{x_1}+(1-\lambda)\mathbf{x_2}\rangle &= \lambda\langle\mathbf{a},\mathbf{x-1}\rangle + (1-\lambda)\langle\mathbf{a},\mathbf{x_2}\rangle \\
        &\leq \lambda b + (1-\lambda)b = b
      \end{align*}
      Therefore, any convex combination of two points in $H$ is still in half-space $H$ so half-spaces are convex.
    \end{enumerate}
  \item[7.4]
    \begin{enumerate}
      \item[(i)]
        \begin{align*}
          \|\mathbf{x}-\mathbf{p}\|^2+\|\mathbf{p}-\mathbf{y}\|^2 + 2\langle\mathbf{x}-\mathbf{p},\mathbf{p}-\mathbf{y}\rangle &= \langle\mathbf{x}-\mathbf{p},\mathbf{x}-\mathbf{p}\rangle + \langle\mathbf{x}-\mathbf{p},\mathbf{p}-\mathbf{y}\rangle \\
          &+ \langle\mathbf{p}-\mathbf{y},\mathbf{p}-\mathbf{y}\rangle + \langle\mathbf{x}-\mathbf{p},\mathbf{p}-\mathbf{y}\rangle \\
          &= \langle \mathbf{x}-\mathbf{p}, \mathbf{x}-\mathbf{p}+\mathbf{p}-\mathbf{y} \rangle \\
          &+ \langle \mathbf{x}-\mathbf{p} + \mathbf{p}-\mathbf{y}, \mathbf{p}-\mathbf{y} \rangle \\
          &= \langle \mathbf{x}-\mathbf{p}, \mathbf{x}-\mathbf{y} \rangle + \langle \mathbf{x}-\mathbf{y},\mathbf{p}-\mathbf{y} \rangle  \\
          &= \langle \mathbf{x}-\mathbf{y}, \mathbf{x}-\mathbf{p}+\mathbf{p}-\mathbf{y} \rangle \\
          &= \langle \mathbf{x}-\mathbf{y},\mathbf{x}-\mathbf{y}\rangle \\
          &= \|\mathbf{x}-\mathbf{y}\|^2
        \end{align*}
      \item[(ii)]
        Suppose we have that $\langle\mathbf{x}-\mathbf{p},\mathbf{p}-\mathbf{y}\rangle \leq 0 \quad\forall \mathbf{y}\in C$. Since the inner product is always positive, we have that:
        \begin{align*}
          \|\mathbf{x}-\mathbf{y}\|^2&=\|\mathbf{x}-\mathbf{p}\|^2+\|\mathbf{p}-\mathbf{y}\|^2 + 2\langle\mathbf{x}-\mathbf{p},\mathbf{p}-\mathbf{y}\rangle \\
          &\geq \|\mathbf{x}-\mathbf{p}\|^2+\|\mathbf{p}-\mathbf{y}\|^2 \\
          &\geq  \|\mathbf{x}-\mathbf{p}\|^2 \\
          \therefore \|\mathbf{x}-\mathbf{y}\| &> \|\mathbf{x}-\mathbf{p}\| \quad \forall \mathbf{y}\in C, \mathbf{y}\neq\mathbf{p}
        \end{align*}
      \item[(iii)]
        Suppose $\mathbf{z}=\lambda\mathbf{y}+(1-\lambda)\mathbf{p}$ for $\lambda\in[0,1]$. Then we have:
        \begin{align*}
          \|\mathbf{x}-\mathbf{z}\|^2 &= \langle \mathbf{x}-\mathbf{z},\ \mathbf{x}-\mathbf{z}\rangle \\
          &= \langle \mathbf{x}-\mathbf{z},\mathbf{x}-\lambda\mathbf{y}-(1-\lambda)\mathbf{p} \rangle \\
          &= \langle \mathbf{x}-\mathbf{z}, \mathbf{x}-\mathbf{p} \rangle + \lambda \langle\mathbf{x}-\mathbf{z}, \mathbf{p}-\mathbf{y} \rangle \\
          &= \langle \mathbf{x}-\mathbf{p},\mathbf{x}-\mathbf{p} \rangle + \lambda \langle \mathbf{p}-\mathbf{y}, \mathbf{x}-\mathbf{p}\rangle + \lambda\langle \mathbf{x}-\mathbf{p},\mathbf{p}-\mathbf{y} \rangle + \lambda^2\langle \mathbf{p}-\mathbf{y},\mathbf{p}-\mathbf{y}\rangle \\
          &= \|\mathbf{x}-\mathbf{p}\|^2+2\lambda\langle \mathbf{x}-\mathbf{p},\mathbf{p}-\mathbf{y} \rangle + \lambda^2\|\mathbf{y}-\mathbf{p}\|^2
        \end{align*}
      \item[(iv)]
        If $\mathbf{p}$ is a projection of $\mathbf{x}$ onto convex set $C$, then by definition $\|\mathbf{x}-\mathbf{p}\| \leq \|\mathbf{x}-\mathbf{y}\| \quad\forall \mathbf{y} \in C$. Since $C$ is convex, $\mathbf{z} = \lambda\mathbf{y}+(1-\lambda)\mathbf{p} \in C \quad\forall \mathbf{y}\in C, \lambda \in[0,1]$ and we have that $\|\mathbf{x}-\mathbf{p}\| \leq \|\mathbf{x}-\mathbf{z}\|$. Thus, $0 \leq \|\mathbf{x}-\mathbf{z}\|-\|\mathbf{x}-\mathbf{p}\| = 2\lambda\langle \mathbf{x}-\mathbf{p},\mathbf{p}-\mathbf{y} \rangle + \lambda^2\|\mathbf{y}-\mathbf{p}\|^2$ from part (iii). Since $0 \leq \lambda$, we have that $0 \leq 2\langle \mathbf{x}-\mathbf{p},\mathbf{p}-\mathbf{y} \rangle + \lambda\|\mathbf{y}-\mathbf{p}\|^2$.
    \end{enumerate}
    $\Longrightarrow$

    Suppose that a point $\mathbf{p}$ is a projection of $\mathbf{x}$ onto convex set $C$. From part (iv), we know that $0 \leq 2\langle \mathbf{x}-\mathbf{p},\mathbf{p}-\mathbf{y} \rangle + \lambda\|\mathbf{y}-\mathbf{p}\|^2 \quad\forall\lambda\in[0,1]$. Then the statement holds true for $\lambda=0$, in which case $0 \leq 2\langle \mathbf{x}-\mathbf{p},\mathbf{p}-\mathbf{y} \rangle\Longrightarrow 0 \leq \langle \mathbf{x}-\mathbf{p},\mathbf{p}-\mathbf{y} \rangle$.

    $\Longleftarrow$

    Suppose we have that $\langle\mathbf{x}-\mathbf{p},\mathbf{p}-\mathbf{y}\rangle \leq 0 \quad\forall \mathbf{y}\in C$. According to part(ii), $\|\mathbf{x}-\mathbf{y}\| > \|\mathbf{x}-\mathbf{p}\| \quad \forall \mathbf{y}\in C, \mathbf{y}\neq\mathbf{p}$. By definition, $\mathbf{p}$ is the projection of $\mathbf{x}$ onto convex set $C$.
  \item[7.6]
    Take any two points $\mathbf{x_1},\mathbf{x_2} \in A=\{\mathbf{x}\in\mathbb{R}^n|f(\mathbf{x}\leq c)\}$. Since $f$ is a convec function, for any $\lambda\in[0,1]$, we have that
    \begin{equation*}
      f(\lambda\mathbf{x_1}+(1-\lambda)\mathbf{x_2}) \leq \lambda f(\mathbf{x_1})+(1-\lambda)f(\mathbf{x_2}) \leq \lambda c + (1-\lambda)c = c
    \end{equation*}
    Therefore, any convex combination of two points in $A$ is still in $A$, so set $A$ is convex.
  \item[7.7]
    Let $f(\mathbf{x}) = \sum_{i=1}^k \alpha_i f_i(\mathbf{x})$ where $\alpha_i\in \mathbf{R}_+, f_i:C\rightarrow \mathbb{R},f_i\text{ convex} \quad \forall 1\leq i \leq k$ and $C$ convex. Then $f:C\rightarrow\mathbb{R}$. For any two points $\mathbf{x_1},\mathbf{x_2}\in C$ and $\lambda\in[0,1]$, we have that
    \begin{align*}
      f(\lambda\mathbf{x_1}+(1-\lambda)\mathbf{x_2}) &= \sum_{i=1}^k \alpha_i f_i(\lambda\mathbf{x_1}+(1-\lambda)\mathbf{x_2}) \\
      &\leq \lambda\sum_{i=1}^k \alpha_i f_i(\mathbf{x_1}) + (1-\lambda)\sum_{i=1}^k \alpha_if_i(\mathbf{x_2}) \\
      &= \lambda f(\mathbf{x_1})+ (1-\lambda)f(\mathbf{x_2}) \\
      \therefore & \quad f(\mathbf{x}) \text{ is a convex function}
    \end{align*}
  \item[7.13]
    Let $f:\mathbb{R}^n\rightarrow \mathbb{R}$ be convex and bounded above by $M$. Suppose there exists two points $\mathbf{x_1}\neq\mathbf{x_2} \in \mathbb{R}^n$ such that $f(\mathbf{x_1})>f(\mathbf{x_2})$. Then for all $\lambda\in[0,1]$, we have that
    \begin{align*}
      f(\mathbf{x_1}) &= f(\lambda\frac{\mathbf{x_1}-(1-\lambda)\mathbf{x_2}}{\lambda}+(1-\lambda)\mathbf{x_2}) \\
      &\leq \lambda f(\frac{\mathbf{x_1}-(1-\lambda)\mathbf{x_2}}{\lambda})+(1-\lambda)f(\mathbf{x_2}) \\
      \frac{f(\mathbf{x_1})-(1-\lambda)f(\mathbf{x_2})}{\lambda} &\leq f(\frac{\mathbf{x_1}-(1-\lambda)f(\mathbf{x_2})}{\lambda}) < M \\
      \therefore \frac{f(\mathbf{x_1})-f(\mathbf{x_2})}{\lambda} + f(\mathbf{x_2}) &< M
    \end{align*}
    However, the last statement is a contradiction since as $\lambda \rightarrow 0$, the expression approaches infinity, contradicting the fact that $f$ is bounded above.
  \item[7.20]
    Note that if $f(\mathbf{x}):\mathbb{R}^n\rightarrow\mathbb{R}$ is affine, then we can express it as $f(\mathbf{x}) = L(\mathbf{x})+c$ so for any $\mathbf{x_1},\mathbf{x_2}\in\mathbb{R}^n$ and $a,b\in\mathbb{R}$ we have that
    \begin{align}
      f(a\mathbf{x_1}+b\mathbf{x_2}) &= L(a\mathbf{x_1}+b\mathbf{x_2})+c \\
      &=aL(\mathbf{x_1})+bL(\mathbf{x_2})+c \\
      \therefore f(a\mathbf{x_1}+b\mathbf{x_2}) &= af(\mathbf{x_1})+bf(\mathbf{x_2})+(1-a-b)c
    \end{align}

    Now let us suppose that $f, -f$ are convex and $f$ is NOT affine. Then there exist $\mathbf{x_1},\mathbf{x_2}\in\mathbb{R}^n$ such that equation (3) does not hold for all $a,b\in\mathbb{R}$. Since $f$ is convex, for any $\lambda\in[0,1]$ we have that:
    \begin{align*}
      f(\lambda\mathbf{x_1}+(1-\lambda)\mathbf{x_2}) &\leq \lambda f(\mathbf{x_1})+(1-\lambda)f(\mathbf{x_2}) \\
      f(\lambda\mathbf{x_1}+(1-\lambda)\mathbf{x_2}) &\geq -\lambda f(\mathbf{x_1})-(1-\lambda)f(\mathbf{x_2})
    \end{align*}
    Similarly, since $-f$ is convex, we also have that:
    \begin{align*}
      -f(\lambda\mathbf{x_1}+(1-\lambda)\mathbf{x_2}) &\leq -\lambda f(\mathbf{x_1})-(1-\lambda)f(\mathbf{x_2})
    \end{align*}
    Therefore, $f(\lambda\mathbf{x_1}+(1-\lambda)\mathbf{x_2}) = \lambda f(\mathbf{x_1})+(1-\lambda)f(\mathbf{x_2})$. This contradicts the fact that $\mathbf{x_1},\mathbf{x_2}$ does not satisfy equation (3) for all scalars in $\mathbb{R}$. Therefore, $f$ is affine.
  \item[7.21]
    $\Longrightarrow$

      Suppose $\mathbf{x^*}$ is a local minimizer for the problem with objective function $f$. Since $\phi$ is strictly increasing, then for all $\mathbf{x}\neq\mathbf{x^*}$ that satisfy the constraints, $f(\mathbf{x^*})\leq f(\mathbf{x}) \Longrightarrow \phi\circ f(\mathbf{x^*})\leq \phi\circ f(\mathbf{x})$ so by definition, $\mathbf{x^*}$ is also a minimizer for the objective function $\phi\circ f$.
      
    $\Longleftarrow$

      Suppose $\mathbf{x^*}$ is a local minimizer for the problem with objective function $\phi\circ f$ but not a local minimzer for the problem with objective function $f$. Then there exists $\mathbf{x_0}$ in the neighborhood of $\mathbf{x^*}$ that satisfies the constraints and $f(\mathbf{x_0}) \leq f(\mathbf{x^*}) \Longrightarrow \phi\circ f(\mathbf{x_0})\leq \phi\circ f(\mathbf{x^*})$, which contradicts the fact that $\mathbf{x^*}$ is a local minimizer for $\phi\circ f$. Thus, $\mathbf{x^*}$ is also a local minimizer for $f$.
\end{enumerate}

\end{document}
