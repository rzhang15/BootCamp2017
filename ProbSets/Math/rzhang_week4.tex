\documentclass[letterpaper,12pt]{article}
\usepackage{array}
\usepackage{threeparttable}
\usepackage{geometry}
\geometry{letterpaper,tmargin=1in,bmargin=1in,lmargin=0.5in,rmargin=0.5in}
\usepackage{fancyhdr,lastpage}
\pagestyle{fancy}
\lhead{}
\chead{}
\rhead{}
\lfoot{\footnotesize\textsl{OSM Lab, Summer 2017, Math PS \#4}}
\cfoot{}
\rfoot{\footnotesize\textsl{Page \thepage\ of \pageref{LastPage}}}
\renewcommand\headrulewidth{0pt}
\renewcommand\footrulewidth{0pt}
\usepackage[format=hang,font=normalsize,labelfont=bf]{caption}
\usepackage{amsmath}
\usepackage{amssymb}
\usepackage{amsthm}
\usepackage{natbib}
\usepackage{upgreek}
\usepackage{setspace}
\usepackage{float,color}
\usepackage[pdftex]{graphicx}
\usepackage{hyperref}
\usepackage{mathrsfs}
\usepackage{dsfont}
\hypersetup{colorlinks,linkcolor=red,urlcolor=blue,citecolor=red}
\theoremstyle{definition}
\newtheorem{theorem}{Theorem}
\newtheorem{acknowledgement}[theorem]{Acknowledgement}
\newtheorem{algorithm}[theorem]{Algorithm}
\newtheorem{axiom}[theorem]{Axiom}
\newtheorem{case}[theorem]{Case}
\newtheorem{claim}[theorem]{Claim}
\newtheorem{conclusion}[theorem]{Conclusion}
\newtheorem{condition}[theorem]{Condition}
\newtheorem{conjecture}[theorem]{Conjecture}
\newtheorem{corollary}[theorem]{Corollary}
\newtheorem{criterion}[theorem]{Criterion}
\newtheorem{definition}[theorem]{Definition}
\newtheorem{derivation}{Derivation} % Number derivations on their own
\newtheorem{example}[theorem]{Example}
\newtheorem{exercise}[theorem]{Exercise}
\newtheorem{lemma}[theorem]{Lemma}
\newtheorem{notation}[theorem]{Notation}
\newtheorem{problem}[theorem]{Problem}
\newtheorem{proposition}{Proposition} % Number propositions on their own
\newtheorem{remark}[theorem]{Remark}
\newtheorem{solution}[theorem]{Solution}
\newtheorem{summary}[theorem]{Summary}
%\numberwithin{equation}{section}
\bibliographystyle{aer}
\newcommand\ve{\varepsilon}
\newcommand\boldline{\arrayrulewidth{1pt}\hline}

\begin{document}

\title{Math Problem Set 4 \\
Open Source Macroeconomics Laboratory Boot Camp}
\author{Ruby Zhang}
\maketitle

\begin{enumerate}
  \item[6.1]
    Given $\mathbf{x},\mathbf{y} \in \mathbb{R}^n, a,b \in \mathbb{R}^n$ and $A \in M_n(\mathbb{R})$, choose $\mathbf{w}\in\mathbb{R}^n$ such that
    \begin{align*}
      \text{minimize }&-e^{-\mathbf{x}^T\mathbf{w}} \\
      \text{subject to }& (A\mathbf{w}-A\mathbf{y}-\mathbf{x})^T\mathbf{w} \leq a \\
      & (\mathbf{y}-\mathbf{x})^T\mathbf{w} = b
    \end{align*}
  \item[6.5]
    Letting the vector $\mathbf{x} = [k\quad m]^T \in \mathbb{R}^2$ denote the amount of knobs and milk cartons respectively, the optimization problem in standard form is choosing $\mathbf{x} \in \mathbb{R}^2$ such that:
    \begin{align*}
      \text{minimize }&-[0.05 \quad 0.07]\mathbf{x}\\
      \text{subject to }&
        \begin{bmatrix}
          3 & 4\\
          1 & 2
        \end{bmatrix}
      \mathbf{x}\leq
      \begin{bmatrix}
       240000 \\
       6000
      \end{bmatrix}
    \end{align*}
  \item[6.6]
    Let us find the Jacobian matrix of $f(x,y)$ to pinpoint the critical points:
    \begin{align*}
      Df(x,y) &= [6xy+4y^2+y \quad 3x^2+8xy+x] \\
      &=[y(6x+4y+1)\quad x(3x+1)]
    \end{align*}
    From solving a basic systems of equations, we have that the points where $Df(x,y)=0$ are: $(0,0), (-\frac{1}{3},0), (0,-\frac{1}{4}), (-\frac{1}{9},-\frac{1}{12})$. To see whether they are local maxima, minima, or points, we must find the Hessian matrix and see if it's positive definite at the critical point:
    \begin{align*}
      D^2f(x,y) &=
      \begin{bmatrix}
        6y & 6x+8y+1 \\
        6x+8y+1 & 8x
      \end{bmatrix} \\
      D^2f(0,0) &=
      \begin{bmatrix}
        0 & 1 \\
        1 & 0
      \end{bmatrix} \\
      D^2f(-\frac{1}{3},0) &=
      \begin{bmatrix}
        0 & -1 \\
        -1 & -\frac{8}{3}
      \end{bmatrix} \\
      D^2f(0,-\frac{1}{4}) &=
      \begin{bmatrix}
        -\frac{3}{2} & -1 \\
        -1 & 0
      \end{bmatrix} \\
      D^2f(-\frac{1}{9},-\frac{1}{12}) &=
      \begin{bmatrix}
        -\frac{1}{2} & -\frac{1}{3} \\
        -\frac{1}{3} & -\frac{8}{9}
      \end{bmatrix}
    \end{align*}
    Based on the definition of positive definite matrices, note that $D^2f(0,0), D^2f(-\frac{1}{3},0), D^2f(0,-\frac{1}{4})$ have negative determinants, so all those points are saddle points. However, det($D^2f(-\frac{1}{9},-\frac{1}{12})$) $> 0$ and $D_{11} < 0$ so $(-\frac{1}{9},-\frac{1}{12})$ is a local maximum.
  \item[6.11]
    Using Newton's method and initial starting point $x_0$, we have that the next point is:
    \begin{align*}
      x_1 &= x_0 - \frac{f'(x_0)}{f''(x_0)} = x_0 - \frac{2ax_0+b}{2a} = -\frac{b}{2a}
    \end{align*}
    Plugging $x_1$ back into $f$, we see that $f'(x_1) = -b+b = 0$ and $f''(x_1) = 2a > 0$. Therefore, $x_1$ is a local minimum (and the unique one since only $x_1$ satisfies $f'(x)=0$).
  \item[6.14]
    Please refer to jupyter notebook rzhang15\textunderscore week4\textunderscore 6.14.ipynb for the implementation of Newton's method.

\end{enumerate}

\end{document}
